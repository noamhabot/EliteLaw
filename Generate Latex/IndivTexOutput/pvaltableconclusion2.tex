% latex table generated in R 3.3.2 by xtable 1.8-2 package
% Wed Mar 14 00:58:37 2018
\begin{table}[ht]
\centering
\begin{tabular}{rrrrrrrrrrr}
  \hline
 & WithLawyersLog & WithoutLawyers & FirmFE & NoFirmFE & Lawyers & FE4 & FE1 & FEYear & NoFE & \textbf{Total} \\ 
  \hline
Lawyers & 0 & 0 & 29 & 3 & 0 & 10 & 10 & 9 & 3 & 32 \\ 
  Lawyers^2 & 0 & 0 & 18 & 0 & 0 & 2 & 2 & 2 & 12 & 18 \\ 
  log(Lawyers) & 40 & 0 & 36 & 4 & 0 & 12 & 12 & 12 & 4 & 40 \\ 
  Leverage & 16 & 11 & 41 & 17 & 0 & 14 & 13 & 11 & 20 & 58 \\ 
  MnA Deal Value & 0 & 1 & 1 & 0 & 0 & 0 & 1 & 0 & 0 & 1 \\ 
  Equity Deal Value & 54 & 44 & 156 & 40 & 0 & 38 & 59 & 47 & 52 & 196 \\ 
  IPO Deal Value & 64 & 63 & 192 & 63 & 0 & 61 & 61 & 59 & 74 & 255 \\ 
  MnA Transactions & 16 & 17 & 47 & 5 & 0 & 15 & 15 & 18 & 4 & 52 \\ 
  Equity Transactions & 61 & 50 & 157 & 65 & 0 & 57 & 52 & 54 & 59 & 222 \\ 
  IPO Transactions & 76 & 82 & 185 & 145 & 0 & 91 & 95 & 93 & 51 & 330 \\ 
  Agg MnA & 16 & 7 & 23 & 18 & 0 & 41 & 0 & 0 & 0 & 41 \\ 
  Agg Equity & 5 & 13 & 39 & 0 & 0 & 39 & 0 & 0 & 0 & 39 \\ 
  Agg IPO & 34 & 36 & 64 & 70 & 0 & 134 & 0 & 0 & 0 & 134 \\ 
  GDP & 0 & 0 & 0 & 0 & 0 & 0 & 0 & 0 & 0 & 0 \\ 
   \hline
\end{tabular}
\caption{On the left, we see the variable name. For each of those, we consider each and every one of their regressions
  that have p-values greater than or equal to 0.05. Out of those, we obtain the regression specifications and keep a counter
  for how many of each type of specification there is. The top of the table (columns) signify which specification has how many counts of high p-values.} 
\end{table}
