\documentclass{article}
\usepackage{subfig}
\usepackage{graphicx}
\usepackage{multirow}
\usepackage{float}
\usepackage[landscape, margin=1in, tmargin=0.5in, bmargin=0.5in]{geometry}
\begin{document}
\begin{center}{\LARGE Elite Law Analysis}
\\
\begin{tabular}{rl}\\Professor Joseph Grundfest, Professor Laurie Hodrick, Noam Habot \\January 2018\end{tabular}\end{center}{\large \textbf{Summary Statistics} }
\newpage
{\large \textbf{Correlations} }
\newpage
{\large \textbf{Correlations with AggM\&A and GDP} }
\newpage
{\large \textbf{Regressions} }
\newpage
{\large \textbf{Regression Performance} }
\newpage
{\large \textbf{Model Averaging} }
\newpage
{\large \textbf{Breakpoint Analysis} }
\newpage

{\large \textbf{Model Selection} }
\begin{figure}[H]
\centering
\begin{tabular}{cccc}
\end{tabular}
\end{figure}
We have also generated this analysis when breaking up the data into 3 or more tiers, but they do not appear to show the differences as clearly as the two-tier models. We conjecture that if as we increase the number of tiers by which we factorize the data, there is a higher proportion of zeros in the resulting lowest tier. This causes hightened sensitivity in the signal of the data and deems the plot uninterpretable.
\newpage
{\large \textbf{More Plots} }
\newpage

\end{document}